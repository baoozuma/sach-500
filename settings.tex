
\usepackage[utf8]{inputenc}
\usepackage{adjustbox}
\usepackage[main=vietnamese,english]{babel}
\usepackage{enumitem}
\usepackage{graphicx}
\usepackage{titletoc}
\usepackage{eso-pic}
\usepackage{tikz}
\usepackage{tcolorbox}
\usepackage{fancyhdr}
\usepackage{listings}
\usepackage{subfig}
\usepackage{wrapfig}
\usepackage[a4paper, left=1in, right=1in, top=0.7in, bottom=1in]{geometry}
\usepackage{setspace}
\usepackage[normalem]{ulem}
\usepackage{amsthm}
\definecolor{dk}{rgb}{0,0,0}
\newcommand{\vocab}[1]{\textbf{\color{black}\sffamily #1}}
\newcommand{\vocabsh}[1]{\textbf{\color{c}\sffamily #1}}
\newcommand{\vocabh}[1]{\textbf{\color{mauve}\sffamily #1}}
\newcommand{\vocabf}[1]{\textit{\sffamily #1}}
\onehalfspacing % Chọn 1.5 khoảng cách giữa các dòng
\usepackage{enumitem}
\usepackage{amssymb} % for \blackcircle
\usepackage{amsmath}
\usepackage[framemethod=TikZ]{mdframed}
\definecolor{bk}{rgb}{0.0, 0.0, 0.61}
\definecolor{p}{rgb}{0.1, 0.1, 0.44}
\definecolor{c}{rgb}{0.55, 0.27, 0.07}
\newtheorem*{lemma}{Bổ đề}
\newtheorem{theorem}{\textcolor{p}{Định lý}}
\usepackage{pifont}
\usepackage{mathtools}
\usepackage{etextools}
\usepackage{ifthen}
\usepackage{hyperref}
\renewcommand*\contentsname{\LARGE MỤC LỤC}
%\renewcommand\familydefault{\sfdefault}

\renewcommand{\thesection}{\LARGE\textcolor{dk}{\arabic{section}.}}
\renewcommand{\thesubsection}{\LARGE\textcolor{dk}{\thesection\arabic{subsection}.}}
\newtheorem{m}{\textcolor{p}{Mệnh đề}}
\newtheorem{baitoan}{\vocab{Bài toán}}[section]

\usepackage{chngcntr}
\newtheorem{g}{\begin{center}\textbf{\textcolor{p}{Bài giải}}\end{center}}
\usepackage[inline]{asymptote}
\newtheorem{btv}{\vocab{Bài toán}}[section]
\definecolor{y}{rgb}{1.0, 0.88, 0.21}
\usepackage{CJKutf8}
\pagestyle{fancy}
\usepackage{multido}

\newcommand{\Pointilles}[1]{%
  \par\nobreak
  \noindent\rule{0pt}{1.5\baselineskip}% Provides a larger gap between the preceding paragraph and the dots
  \multido{}{#1}{\noindent\makebox[\linewidth]{\dotfill}\endgraf}% ... dotted lines ...
  \bigskip% Gap between dots and next paragraph
}
% Redefine \Leftrightarrow to \Lra
\let\lra\Leftrightarrow
\let\Leftrightarrow\relax
\newcommand{\fra}{\forall}
\newcommand{\xyr}{\forall x,y \in \bb{R}}
\newcommand{\xr}{\forall x\in \bb{R}}
\newcommand{\yr}{\forall y\in \bb{R}}
\newcommand{\bb}{\mathbb}
\newcommand{\dbi}{\displaystyle\binom}
\newcommand{\dsum}{\displaystyle\sum}
\newcommand{\dlim}{\displaystyle\lim_{n \to +\infty}}
\newcommand{\dprod}{\displaystyle\prod}
\newcommand{\dint}{\displaystyle\int}
\newcommand{\csum}{\dsum_{cyc}}
% Redefine \Rightarrow to \Ra
\let\ra\Rightarrow
\let\Rightarrow\relax
\definecolor{dkgreen}{rgb}{0,0.6,0}
\definecolor{gray}{rgb}{0.5,0.5,0.5}
\definecolor{mauve}{rgb}{0.58,0,0.82}
\newcommand{\DeclarePairedDelimiterCase}[2]{%
\newcommand#1[1][]{%
	\ifthenelse{\equal{##1}{normal}}%
	{#2}%
	{%
		\ifthenelse{\equal{##1}{big}\OR\equal{##1}{Big}\OR\equal{##1}{bigg}\OR\equal{##1}{Bigg}}%
		{\expandnext{#2[}{\csname##1\endcsname}]}%
		{#2*}%        % standard case using \left and \right
	}%
}%
}
\newcommand{\DeclarePairedDelimiterY}[4][Temp]{%
\expandafter\DeclarePairedDelimiter\csname#2#1\endcsname{#3}{#4}%
\expandnext{\expandnext{\DeclarePairedDelimiterCase}{\csname#2\endcsname}}{\csname#2#1\endcsname}%
}
\newcommand{\DeclarePairedDelimiterXY}[6][Temp]{%
\expandafter\DeclarePairedDelimiterX\csname#2#1\endcsname[#3]{#4}{#5}{#6}%
\expandnext{\expandnext{\DeclarePairedDelimiterCase}{\csname#2\endcsname}}{\csname#2#1\endcsname}%
} 
\DeclarePairedDelimiter\floor{\lfloor}{\rfloor}

\lstset{
  frame=none, % Xóa khung ở trên và dưới
  language=Python,
  aboveskip=3mm,
  belowskip=3mm,
  showstringspaces=false,
  columns=flexible,
  basicstyle={\small\ttfamily},
  numbers=none,
  keywordstyle=\color{blue},      % Màu sắc cho các từ khóa
  commentstyle=\color{green!60},  % Màu sắc cho các comment
  stringstyle=\color{brown},        % Màu sắc cho chuỗi
  numberstyle=\tiny\color{gray},  % Màu sắc cho số dòng
  breaklines=true,
  breakatwhitespace=true,
  tabsize=3
}
\renewcommand{\headrulewidth}{0.005em}

\makeatother
\makeatletter

\fancypagestyle{plain}{%
	\fancyhf{}%
	\fancyfoot[C]{\thepage}
	\renewcommand{\headrulewidth}{0pt}%
}

\newcommand{\circled}[1]{\tikz[baseline=(char.base)]{
		\node[shape=circle,fill,inner sep=3pt,text=white] (char) {\footnotesize #1};}}
\renewcommand{\arraystretch}{1.5}
\let\oldfrac\frac
\renewcommand{\frac}[2]{\displaystyle\oldfrac{#1}{#2}}
% Định nghĩa môi trường mới
\newmdenv[linecolor=black, linewidth=0.02cm]{myframe}
\newmdenv[linecolor=bk, backgroundcolor=blue!4,  linewidth=0.04	cm]{fr}
% Môi trường "bài toán"
\newenvironment{bt}
{
	\color{p}\begin{myframe}
		\begin{baitoan}\upshape
		}
		{
		\end{baitoan}
	\end{myframe}
}
\newenvironment{thr}
{
	\color{p}\begin{fr}
		\begin{theorem}\upshape
		}
		{
		\end{theorem}
	\end{fr}
}
\newenvironment{md}
{\m\upshape}
{\endm}
\newenvironment{btvn}
{
	
		
		\begin{btv}\upshape
		}
		{
		\end{btv}

}
\renewcommand\proofname{}
\newenvironment{sol}{%
    \begin{center}%
        \textbf{\vocab{Bài giải}}%
	\end{center}%
        \begin{proof}[\unskip\nopunct]%
}{%
        \end{proof}%
    
}
\newenvironment{pro}{%

	\textbf{\vocab{Chứng minh}}%

	\begin{proof}[\unskip\nopunct]%
}{%
	\end{proof}%

}

\renewcommand\them{\textcolor{p}{\arabic{m}}}
\renewcommand\thebtv{\vocab{\arabic{btv}}}

\newcommand{\p}[1]{\textbf{\textcolor{p}{#1}}}
\newcommand{\pf}[1]{\textit{\textcolor{p}{#1}}}
\renewcommand\thebaitoan{\vocab{\arabic{baitoan}}}
\newcounter{theo}[section]\setcounter{theo}{0}
\renewcommand{\thetheo}{\arabic{section}.\arabic{theo}}
\newenvironment{theo}[2][]{%
	\refstepcounter{theo}%
	\ifstrempty{#1}%
	{\mdfsetup{%
			frametitle={%
				\tikz[baseline=(current bounding box.east),outer sep=0pt]
				\node[anchor=east,rectangle,fill=yellow]
				{\strut };}}
	}%
	{\mdfsetup{%
			frametitle={%
				\tikz[baseline=(current bounding box.east),outer sep=0pt]
				\node[anchor=east,rectangle,fill=yellow!70, draw=black, line width=0.03cm]
				{\strut \vocab{#1}};}}%
	}%
	\mdfsetup{innertopmargin=5pt,linecolor=black,%
		linewidth=0.03cm,topline=true,%
		frametitleaboveskip=\dimexpr-\ht\strutbox\relax, 
	}
	\begin{mdframed}[]\relax%
		\label{#2}}{\end{mdframed}}
	
\newcounter{lem}[section]\setcounter{lem}{0}
\renewcommand{\thelem}{\arabic{section}.\arabic{lem}}
\newenvironment{lem}[2][]{%
	\refstepcounter{lem}%
	\ifstrempty{#1}%
	{\mdfsetup{%
			frametitle={%
				\tikz[baseline=(current bounding box.east),outer sep=0pt]
				\node[anchor=east,rectangle,fill=yellow]
				{\strut };}}
	}%
	{\mdfsetup{%
			frametitle={%
				\tikz[baseline=(current bounding box.east),outer sep=0pt]
				\node[anchor=east,rectangle,fill=yellow]
				{\strut \vocab{#1}};}}%
	}%
	\mdfsetup{linecolor=black,%
		linewidth=0.03cm,topline=true,%
		frametitleaboveskip=\dimexpr-\ht\strutbox\relax
	}
	\begin{mdframed}[]\relax%
		\label{#2}}{\end{mdframed}}
		\newcommand{\fa}{%
		\begin{tikzpicture}[x=1.4ex, y=1.4ex, scale=0.5, baseline=-.8ex]
			\draw (-1,1)  -- (-1,0) -- (1,0) -- (1,-1);
			\draw (-1,-1) -- (0,-1) -- (0,1) -- (1,1);
		\end{tikzpicture}%
	}
	
	\newcommand{\func}[1]{f\left(#1\right)}
